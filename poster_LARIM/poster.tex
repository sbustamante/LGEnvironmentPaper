%*************************************************************************
%
% This is a LaTeX file for an A3 poster.
%
%*************************************************************************


\documentclass{article}
% To modify the size of the page:
\usepackage[dvips,a3paper,landscape,centering,margin=1cm]{geometry}
%\geometry{papersize={110cm,150cm}}
\usepackage{multicol}
\usepackage[latin1]{inputenc}
\usepackage{color}

\usepackage{amsmath, amsthm, amsfonts}
\usepackage{graphicx}           % Include figure files.

% Colors
% -------
\definecolor{light_blue}{rgb}{0.8,0.85,1}
\definecolor{marronrp3}{rgb}{.9,.9,.7}
\definecolor{salmon}{rgb}{1,.9,.8}
\definecolor{rojo}{rgb}{.6,.1,0}

\pagestyle{empty}

\def\to{\rightarrow}

% ===========================================================================

\title{}
\author{}
\date{}

\begin{document}
%\maketitle

\begin{center}
  \begin{minipage}{.15\linewidth}
    \includegraphics[width=.6\linewidth]{LARIM_logo.png}
  \end{minipage}
  %&
  \begin{minipage}{.56\linewidth}
    \begin{center}
      \Huge \textbf{The Place of the Local Group in the Cosmic Web}

  \vspace{.01\linewidth}      
  \small
  
  \begin{minipage}{.12\linewidth}
    \includegraphics[width=0.5\linewidth]{UdeA_logo.png}
  \end{minipage}
  \begin{minipage}{0.25\linewidth}
    \begin{flushright}
      \textbf{Sebasti�n Bustamante Jaramillo}\\
      Instituto de F�sica\\
      Universidad de Antioquia, Colombia\\
    \end{flushright}
  \end{minipage}
    \hspace{.05\linewidth}
    \begin{minipage}{.12\linewidth}
    \includegraphics[width=1.0\linewidth]{UniAndes_logo.png}
  \end{minipage}
  \begin{minipage}{0.25\linewidth}
    \begin{flushright}
      \textbf{Jaime Forero-Romero, Ph.D}\\
      Departamento de F�sica\\
      Universidad de los Andes, Colombia\\
    \end{flushright}
  \end{minipage}
    \end{center}
  \end{minipage}
\end{center}
  \colorbox{marronrp3}{
  \begin{minipage}[t]{0.985\linewidth}
  \centering
  \small
The influence of cosmological environment on the properties of large-scale structures has been 
extensively discussed in previous works, following this direction we aim to determinate a possible
influence on the kinematic properties of our Local Group of Galaxies. We use the Bolshoi simulation 
in the current time (z=0), with a comovil volume of 256 Mpc$^3 h^3$ and a BDM catalog of dark matter 
halos and subhalos systems. Quantification of the environment is made using the velocity tensor 
scheme (Vweb) and the potential tensor scheme (Tweb). We detect LG like systems using gravitational
isolations criteria and imposing observational constrains in the halos pair systems. To build a 
more faithful sample of LG systems we detect large-scale void regions using a FOF scheme and
select a subsample of LG systems near to those void regions. Finally we calculate the kinematic 
properties of each sample to determinate correlations with their environment selection criteria.
We find a significative bias in the radial velocity and the specific energy of LG systems, whereas 
the specific angular momentum and the tangential velocity don't seem to have important bias.
  \end{minipage}
} 

\vspace{.1cm}

% ---------------------------------------------------------------------------

\setlength{\columnsep}{1cm}
\begin{multicols}{3}

\section*{Motivation}

The coagulation-fragmentation equations describe the
evolution of a large number of clusters which can stick together or
break. Here we deal with the discrete version.

\begin{center}

  \begin{tabular}{lll}
    $c_j$  &  $\equiv$  &  number density of\\
    &&                     clusters of size j
  \end{tabular}
  \vspace{.2cm}

  % Thus, $c_j$ can be measured, for instance, in clusters per cubic
  % meter, in units of $\unit{m^{-3}}$.

  \begin{minipage}[t]{.4\linewidth}
    \begin{center}
      % -------------------------------------
%      \includegraphics[width=3cm]{eps}
      % -------------------------------------

      \begin{tabular}{cc}
        $b_{jk}$  $\equiv$ &  rate of occurrence of\\
        &  reaction $j+k \to j,k$
      \end{tabular}
    \end{center}
  \end{minipage}
  % 
  \hspace{.3cm}
  % 
  \begin{minipage}[t]{.4\linewidth}
    \begin{center}
      % ------------------------------------
%      \includegraphics[width=3cm]{coag.eps}
      % ------------------------------------

      \begin{tabular}{cc}
        $a_{jk}$   $\equiv$ &  rate of occurrence of\\
        &  reaction $j \to j+k$
      \end{tabular}
    \end{center}                    
  \end{minipage}

\end{center}

% ---------------------------------------------------------------------------

\vspace{.2cm}

\noindent
\colorbox{marronrp3}{
  \begin{minipage}[t]{.96\linewidth}
    \Large
    \begin{align*}
      \frac{d}{dt} c_j
      & = &&  \frac{1}{2} \sum_{k=1}^{j-1} a_{k,j-k}  c_k c_{j-k}
      &  \text{Coagulation gain}\\
      && - &\sum_{k=1}^{\infty} a_{jk} c_j c_k
      &   \text{Coagulation loss}\\
      && + &\sum_{k=j+1}^{\infty} b_{j,k-j} c_k
      &   \text{Fragmentation gain}\\
      && - &\frac{1}{2} \sum_{k=1}^{j-1} b_{k,j-k} c_j
      & \text{Fragmentation loss}
    \end{align*}
    \vspace{.02cm}
  \end{minipage}
}
\vspace{.4cm}

{\textcolor{rojo}{The generalized Becker-D�ring system is the special
    case where $a_{jk}$ and $b_{jk}$ are zero whenever $\min\{j,k\} >
    N$ for some $N$. For $N=1$ the system is the Becker-D�ring
    system.}  }

\section*{Asymptotic Behavior}

The study of the long-time behavior of solutions to these equations is
expected to be a model of physical processes such as phase transition.
Under certain general conditions which include a detailed balance we
can ensure the existence of equilibrium states. In these conditions,
there is a critical mass $\rho_s \in ]0,\infty[$ such that any
solution that initially has mass $\rho_0 \leq \rho_s$ will converge
for large times, in a certain strong sense, to an equilibrium solution
with mass $\rho_0$. On the other hand, any solution with mass above
$\rho_s$ converges (in a weak sense) to the only equilibrium with mass
$\rho_s$; this weak convergence can then be interpreted as a phase
transition in the physical process modelled by the equation.

Convergence in this weak sense means that a fixed part of the total
mass of particles is found to be forming larger and larger clusters as
time passes and the mean size of clusters goes to infinity. The
physical interpretation of this, depending on the context, can be a
change of phase or the apparition of crystals, for example.

\noindent
\begin{center}
  \noindent
  \colorbox{marronrp3}{
    \begin{minipage}[t]{.96\linewidth}
      \begin{align*}
        & \text{\Large Below critical mass}
        &\to \quad
        &\begin{cases}
          \text{ \Large Trend to equilibrium }\\
          \text{ \Large Strong convergence }
        \end{cases}
        &
        \\
        &\text{\Large Over critical mass }
        &\to \quad
        &\begin{cases}
          \text{ \Large Large clusters created}\\
          \text{\Large Weak convergence }
        \end{cases}
        &
      \end{align*}
    \end{minipage}
  }
\end{center}

\begin{center}  

\vspace{.5cm}

\Large
\begin{tabular}[t]{c|c}
  \multicolumn{2}{c}{\huge \textbf{Previous results}}
  \vspace{.3cm}
  \\
  Becker-D�ring& Ball, Carr, Penrose\\
  system       & \cite{BCP86,BC88} (1986-88)\\
  \hline
  Generalized Becker-D�ring &Carr, da Costa\\
  (rapidly decaying initial data) &\cite{CdC94} (1994)\\
  \hline
  Generalized Becker-D�ring & da Costa \\
  (small initial data) & \cite{dC98} (1998)
\end{tabular}
\end{center}


% ---------------------------------------------------------------------------
  
\section*{Sketch of the proof}

Our proof is a generalization of a method used in unpublished notes by
Ph. Lauren\c{c}ot and S. Mischler \cite{LM}, inspired by the proof of
uniqueness of solutions to the Becker-D\"oring equation in
\cite{LM02e}.

It is known that, under common assumptions, \emph{there is always} at
least weak convergence to a certain equilibrium state;
\textcolor{rojo}{the problem reduces to show that for an initial
  density under the critical one solutions converge \emph{strongly} to
  the equilibrium \emph{with the same density}}. To prove this, it is
enough to show that the tails of the solutions are small enough, so
that strong convergence holds. The following estimate, roughly stated
here, is the key of our proof:

\noindent
\colorbox{marronrp3}{
  \begin{minipage}[t]{.96\linewidth}
    \vspace{.2cm}
    \centerline{\huge \textbf{Main estimate}}
    \vspace{.05cm}

    \Large
    If $c = \{c_j\}_{j \geq 1}$ is a solution to the generalized
    Becker-D�ring equations with density below the critical one, then
    there is some sequence $r_i$ (which tends to zero as $i \to
    \infty$) such that the tails of the solution have mass below
    $r_i$; this is,
    \begin{equation*}
      \sum_{k=i}^\infty k c_k(t) \leq r_i
    \end{equation*}
    for all times $t$ after some time $t_0$.
    \\\hspace{.05cm}
  \end{minipage}
}

\vspace{.5cm}

The proof of this consists mainly of an estimate obtained by
differentiating the quantity $H_i := (G_i-r_i)_+$ (the positive part
of $G_i - r_i$), proving with a differential inequality that it must
remain zero for all times starting from a certain $t_0$.
% ---------------------------------------------------------------------------
%
\small
\begin{thebibliography}{}

\bibitem{BCP86} J. M. Ball, J. Carr, O. Penrose, \emph{The
    Becker-D\"oring cluster equations: basic properties and asymptotic
    behaviour of solutions}, Comm. Math. Phys. 104, 657--692 (1986)

\bibitem{BC88} J. M. Ball, J. Carr, \emph{Asymptotic behaviour of
    solutions to the Becker-D\"oring equations for arbitrary initial
    data}, Proc. Roy. Soc. Edinburgh Sect. A, 108, 109-116 (1988)
  
\bibitem{C04} J. A. Ca�izo, \emph{Asymptotic behavior of solutions to
    the generalized Becker-D�ring equations for general initial data},
  preprint.
  
\bibitem{CdC94} J. Carr, F. P. da Costa, \emph{Asymptotic behaviour of
    solutions to the coagulation-fragmentation equations. II. Weak
    fragmentation}, J. Stat. Phys. 77, 89--123 (1994)

\bibitem{dC98} F. P. da Costa, \emph{Asymptotic behaviour of low
    density solutions to the generalized Becker-D\"oring equations},
  NoDEA Nonlinear Differential Equations Appl. 5, 23--37, (1998)
  
\bibitem{LM} Ph. Lauren\c{c}ot, S. Mischler, \emph{Notes on the
    Becker-D\"oring equation}, personal communication.

\bibitem{LM02e} Ph. Lauren{\c{c}}ot, S. Mischler, \emph{From the
    {B}ecker-{D}\"oring to the {L}ifshitz-{S}lyozov-{W}agner
    equations}, J. Statist. Phys. 106, 5-6, pages 957--991 (2002).

\end{thebibliography}


\end{multicols}

\end{document}