\documentclass[usenatbib]{latex/mn2e} 
%External Packages and personalized macros
\include{latex/macros}

\begin{document}

%=========================================================================
%		FRONT MATTER
%=========================================================================
\title[LG Environment]{The place of the local group in the cosmic web}
\author[S. Bustamante and J.E. Forero-Romero]{
\parbox[t]{\textwidth}{\raggedright 
  Sebastian Bustamante \thanks{sbustama@pegasus.udea.edu.co}$^{1}$ 
  Jaime E. Forero-Romero$^{2}$ 
}
\vspace*{6pt}\\
$^1$Instituto de F\'{\i}sica - FCEN, Universidad de Antioquia, Calle
67 No. 53-108, Medell\'{\i}n, Colombia\\ 
$^2$Departamento de F\'{i}sica, Universidad de los Andes, Cra. 1
No. 18A-10, Edificio Ip, Bogot\'a, Colombia
}

\maketitle

\begin{abstract}


We present here a study about the influence of the environment on the
local group (LG) of galaxies in the context of $\Lambda$CDM. In this study 
we use a large volume high resolution N-body cosmological simulation 
(Bolshoi) together with the most recent methods to quantify the cosmic web 
(T-web, V-web schemes); furthermore we propose a novel approximation, base
upon the minimization of the mean density of void regions, to determinate 
the optimum threshold value $\lambda_{th}$, which have been treated until 
now as a free parameter. 
Following the recent work of Courtois et al. (2013), where was 
found that the LG is located near to a large void, we also do an extensive 
study of voids, applying a FOF algorithm to find void regions and 
performing an analysis of their shape based upon the reduced inertia 
tensor.
Using the recent observations that constrained the tangential velocity of
M31 with respect to the Milky Way (MW), the previously established radial 
velocity, the estimated masses of dark halos, along with some criteria
to guarantee the gravitational isolation of these systems (Forero-Romero
et al. 2013-1), we select a set of halos pairs as a representative sample 
of LG systems in the Bolshoi simulation. 
We look for possible bias and correlations between the environment 
properties of each LG system and its kinematic and formation properties. 
Among our main result we find \SRKED{[summarize our results here!]}.

\end{abstract}

\begin{keywords}
Cosmology: large-scale Structure of Universe, 
galaxies: star formation - line: formation
\end{keywords}


%=========================================================================
%		PAPER CONTENT
%=========================================================================

%*************************************************************************
\section{Introduction}
\label{sec:introduction}
%*************************************************************************


The spatial distribution of galaxies describes a web-like pattern, the 
so-called cosmic web. Today it is understood that such configuration is 
driven by gravitational instabilities. ...



The study of the influence of the cosmic web on galaxy properties start 
with the seminal work of Dressler \SRKED{[reference here]} and extends to 
recent works using large observational surveys that look for signatures of 
the web into the evolution of galaxy populations. With the advention of 
more detailed observations and sophisticated computational models it is 
now within our reach to understand what physical processes dominate.



This makes  that the mass assembly history of a galaxy is deeply connected 
with its  position in the cosmic web. There is an extensive body of 
literature on the effects of the web environment on the observable 
properties of galaxies. 



This environmental study is also of paramount importance to understand the 
formation of our Galaxy. In our local neighborhood, the observations of 
dwarf galaxies around the Milky Way (MW) and the Andromeda galaxy (M31) 
show filamentary and disk-like patterns that can be linked to a 
preferential infall direction, very likely connected with the cosmic web 
where the Local Group (LG) of galaxies is embedded. 



In this paper we quantify the velocity shear environment of DM halo pairs
representative of the principal members of the Local Group (LG), the Milky
Way (MW) and Andromeda galaxy (M31). We perform this study in an 
unconstrained cosmological simulation from random phases in the initial 
conditions, and unlike previous works, where were used constrained 
cosmological simulations which have been setup as to reproduce the large 
scale structure of the local universe, we use directly observational 
measurements of the kinematics properties of the local group \SRKED{
[Reference here]} in order to build faithful samples of LG-like systems.



We pay special attention to the correlation of the present velocity shear 
environment with the assembly and the kinematics properties of the pairs. 
The motivation to have that focus is that it has been previously shown 
that the LG present in three different realizations of the constrained 
simulations have assembly histories biased towards early formation times 
and absence of major mergers (ratio 1:10) in the last $10$ Gyr. In the 
case of the kinematic properties, recent observational constrains to the 
galactocentric tangential velocity of M31 has enabled to establish how 
typical is the LG in a cosmological context \SRKED{[reference to 
Forero-Romero et.al 2013-1]}, that is why we focus here how a specific kind 
of host environment biases these kinematics properties.



%*************************************************************************
\section{The Simulation}
\label{sec:the_simulation}
%*************************************************************************


As it was previously mentioned, we use an unconstrained cosmological 
simulation, the Bolshoi simulation, to identify the possible large scale 
environment of the Local Group. This is a similar approach to the one already 
used by \SRKED{[reference here]}.



The Bolshoi simulations follows the non-linear evolution of a dark matter 
density field on a cubic volume of size $250$\hMpc sampled with $2048^3$ 
particles. The cosmological parameters in the simulation are 
$\Omega_{\rm m}=0.27$, $\Omega_{\Lambda}=0.73$, $h=0.70$, $n=0.95$ and 
$\sigma_{8}=0.82$ for the matter density, cosmological constant, 
dimensionless Hubble parameter, spectral index of primordial density 
perturbations and normalization for the power spectrum. The mass of each 
particle in the simulation is $m_{\rm p}=1.4\times 10^{8}$\hMsun.



%-------------------------------------------------------------------------
\subsection{Halos and Merger Trees}
\label{subsec:halos_merger_trees}
%-------------------------------------------------------------------------



We identify halos with two algorithms, the Friends-of-Friends \SRKED{
[reference here]} algorithm and the Bound Density Maximum algorithm. The 
constructed catalogues also provide the basis for the mass aggregation 
history studies. We also include in the catalogues information about the 
substructure.



All the results presented here must be interpreted in term of host halos, 
without any information of the substructure. In particular the merger of 
two FOF halos corresponds to the epoch of first overlap, and not to the 
fusion and/or disruption of an accreted sub-halo with a dominant halo. 



The linking length is $b=0.17$ times the mean inter-particle separation. 
All objects with 20 particles or more are considered a bona fide halo and 
are included in the construction of the merger tree, this corresponds to a 
minimum halo mass of $M_{\rm min}=2.70\times 10^{9}$\hMsun in the Bolshoi 
simulation.



The halo identification for the simulation was done for XX snapshots in 
the redshift range $0<z<7$ more or less evenly spaced in look-back time.



%*************************************************************************
\section{Algorithms to quantify the cosmic web}
\label{sec:algorithms_cosmic_web}
%*************************************************************************



%-------------------------------------------------------------------------
\subsection{The tidal web (T-web)}
\label{subsec:Tweb}
%-------------------------------------------------------------------------



The first algorithm  we use to identify the cosmic web is based upon the
diagonalization of the tidal tensor, defined as the Hessian of a 
normalized gravitational potential  


%.........................................................................
%Tidal Tensor
\begin{equation}
T_{\alpha\beta} = \frac{\partial^2\phi}{\partial x_{\alpha}\partial x_{\beta}}
\end{equation}
%.........................................................................
where the physical gravitational potential has been rescaled by a factor 
$4\pi G\bar{\rho}$ in such a way that $\phi$ satisfies the following 
equation



%.........................................................................
%Poisson
\begin{equation}
\nabla^2\phi = \delta,
\end{equation}
%.........................................................................
where $\bar{\rho}$ is the average density in the Universe, $G$ is the 
gravitational constant and $\delta$ is the dimensionless matter 
overdensity.



%-------------------------------------------------------------------------
\subsection{The velocity  web (V-web)}
\label{sec:Vweb}
%-------------------------------------------------------------------------



We also use a kinematical method to define the cosmic-web environment in 
the simulation. The method has been thoroughly described in XXX and 
applied to study the shape and spin alignment in the Bolshoi simulation 
here XX. We refer the reader to these papers to find a detailed 
description of the algorithm, its limitations and capabilities. Here we 
summarize the most relevant points for the discussion. 



The V-web method for environment finding is based on the local shear 
tensor calculated from the smoothed DM velocity field in the simulation.
The central quantity is the following dimensionless quantity 


%.........................................................................
%V-Web Definition
\eq{V_web}
{
\Sigma_{\alpha\beta} = -\frac{1}{2H_0}\pr{\frac{\partial v_{\alpha}}
{\partial x_{\beta}}+\frac{\partial v_{\beta}}{\partial x_{\alpha}}}
}
%.........................................................................
where $v_{\alpha}$ and $x_{\alpha}$ represent the $\alpha$ component of 
the comoving velocity and position, respectively. $\Sigma_{\alpha\beta}$ 
can be represented by a $3\times 3$ symmetric matrix with real values,
that ensures that is possible to diagonalize and obtain three real 
eigenvalues $\lambda_{1} > \lambda_{2}>\lambda_3$ whose sum (the trace of
$\Sigma_{\alpha\beta}$) is proportional to the divergence of the local 
velocity field smoothed on the physical scale ${\mathcal R}$. 



The relative strength of the three eigenvalues with respect to a threshold
value $\lambda_{th}$ allows for the local classification of the matter 
distribution into four web types: voids, sheets, filaments and peaks, 
which correspond to regions with 3, 2, 1 or 0 eigenvalues with values 
larger than $\lambda_{th}$. Below we shall discuss a novel approach to 
define an adequate threshold value based on the visual impression of void
regions, furthermore we study other possible values based on other visual
features of the cosmic web.



%-------------------------------------------------------------------------
\subsection{The cosmic web in Bolshoi}
\label{subsec:web_in_simulations}
%-------------------------------------------------------------------------



Both established schemes to quantify the cosmic web depend on continuous 
and smooth physical quantities, as the peculiar velocity field and the 
density field. In order to calculate the necessary tensors, a discretization
of the simulation volume is performed, so all the properties are reduced 
to single values associated to discrete cells. According to this, we divide 
the overall volume into $(256)^3$ cells, so each cell has an associated 
comovil cubic volume of $0.98 \mbox{ Mpc h}^{-1}$. Finally, to reduce 
possible effects due to the discretization process, a gaussian softening 
is performed between neighbour cells.



Once defined the numerical details about the classification schemes, we 
shall analyse the dependence on the threshold value $\lambda_{th}$ for each 
one. In the figure \ref{fig:mean_density} we show the variation of the mean 
density parameter $\delta$ with the threshold value for cells marked in 
each of the adopted types of environment.



As was previously established by \SRKED{Hoffman et al. (2012)} and as 
can be seen in the figure \ref{fig:mean_density}, the behaviour of the 
V-web scheme is significantly more sensible to variations of the 
$\lambda_{th}$ value compared with the T-web scheme; nevertheless, the 
behaviour of the mean density parameter for voids, sheets and filaments, 
are qualitatively quite similar for the two schemes, reaching extreme 
values in the marked $\lambda_{th}$ critical values respectively. For 
instance, voids reach a minim value at some critical threshold, increasing 
for higher threshold values, while sheets and filaments reach a maxim value
at other critical threshold, decreasing for higher threshold values. 
Although we shall focus our analysis on voids because they are completely 
dominant in the visual impression of the cosmic web, an analogous analysis 
might be performed for other type of environments.



On cosmic scales, the presence of highly non-linear structures implies the 
existence of very vast regions with density lower than the mean 
cosmological value due to the mass conservation. That is why the visual 
impression of the cosmic web must be necessarily dominated by these 
under-dense regions. Keeping that in mind, our novel proposal is based upon
the correct quantification of these regions, so the optimum threshold 
value must be chosen such that: sheet regions do not invade real void 
regions (in such case, the mean density parameter of sheet regions would 
become negative) and void regions do not invade real sheet and filament 
regions (in such case, the mean density parameter of void regions would 
increase due to the contribution of over-dense regions). Thus, the optimum 
value is simply where the mean density parameter of void regions is 
minimized. According to this, we obtain the next threshold values 
for the T-web and the V-web respectively, $\lambda_{th}^T = 0.61$ and 
$\lambda_{th}^V = 0.26$. To verify our analysis, we show in the figure 
\ref{fig:visual_impression} the visual impression for each defined critical 
value, and as can be seen, the chosen values reproduce properly the 
expected impression according to the density field. 



Our classification scheme may be thought as a refinement of the recent 
schemes, where the threshold value is used to taking as a free parameter, 
based on the classic methods, where the classification is performed 
based on a cut off of the density field directly \SRKED{[references here]}.
So we make use of the objectivity achieved by the analysis of the mean 
density, but keeping all the environmental information provided by the
tensorial schemes instead of the poor description provided by the density 
field.



%-------------------------------------------------------------------------
\subsection{Method to find void regions}
\label{subsec:method_voids}
%-------------------------------------------------------------------------



Following the recent work of \SRKED{Courtois et al. 2013}, we use a method 
based on a FOF algorithm to find extended regions of voids in order to
select halo systems according to the proximity to those regions. 
To achieve this, we build the input catalogue of the FOF method with the 
positions of the center coordinate of every cell marked as void according 
to the web scheme adopted; furthermore we set an adequate linking length 
to connect even diagonal neighbour cells.



Following the work of \SRKED{Forero-Romero et al. 2008}, we also perform a 
percolation analysis in order to select the best threshold parameter that
reduce percolation in cells, thereby accounting for physical void regions.
In the figure \ref{fig:percolation_analysis} we show the obtained result 
of our percolation analysis for both web schemes. In both cases it can be 
noted the volume of the largest void region is minimized and the volume 
distribution of voids is relatively flat at $\lambda_{th} = 0.0$, that 
means percolation is completely reduced for this threshold value. So 
despite of the previously established $\lambda_{th}$ optimum values for 
each scheme, we shall use $\lambda_{th} = 0.0$ just for the detection of 
void regions. Besides, due to the domination of the large scale visual 
impression by voids, it is inevitable the presence of percolation 
phenomenon, so the current chosen threshold value for percolation is 
justified because in spite of voids are necessarily connected, we are just 
interested in detecting bulk regions.



Next, we shall calculate the reduced inertia tensor of each void region 
in order to determinate their principal direction of inertia and analyse 
the size-shape distribution of voids.


%.........................................................................
%Reduced inertia tensor
\eq{ReducedIntertia}
{ \tau_{ij} = \sum_l \frac{ x_{l,i}x_{l,j}  }{R_l^2} }
%.........................................................................
where $l$ is a index associated to each cell of to the current region, 
$i$ and $j$ indexes run over each spatial direction and finally 
$R_l$ is defined as $R_l^2 = x_{l,1}^2 + x_{l,2}^2 + x_{l,3}^2$, all 
positions are measured from the respective center of mass of the region.




%*************************************************************************
\section{Local Group Sample Definition}
\label{section:Def_Samples}
%*************************************************************************



In order to establish an adequate set of criteria to define LG samples in 
unconstrained simulations, we proceed from the the general dark halo 
catalogues, constructed using the FOF scheme with a linking length of 
$b=0.17$. These samples are defined for all simulation and will be 
referred as General Halos (GH) samples. To be consistent with the 
observationally determined mass range of halos that host disk-form 
galaxies, we select the halos in the mass range $5\times10^{11} <
M_{h}/\hMsun < 5\times 10^{12}$, referred here as Individual Halos (IH) 
samples. 



As a primal approach to define gravitational bounded halo pairs we select 
all halo pairs in IH samples that satisfied to be the closest to each 
other, they constitute the Pairs (P) samples. To keep the concordance with
previous works in algorithms for LG selection (\SRKED{[references here}), 
we define here the next list of conditions that have to fulfill a pair 
system in order to construct the Isolated Pairs (IP) samples. All these 
considerations are based upon the relative dynamics of the Milky Way and 
M31, and its isolation from massive structures:



%.........................................................................
%Isolated pairs criteria
\begin{enumerate}

\item{The distance between the center of the two halos should be less than 
$0.7$\hMpc.}

\item{The relative physical velocity between the two halos has to be 
negative.}

\item{The distance to any halo more massive than any of the pair members 
must be less than $2$\hMpc.}

\item{The distance to cluster-like halos with masses larger than 
$1\times10^{13}$ \hMsun must be larger than $5$\hMpc.}
\end{enumerate}
%.........................................................................


In the case of a constrained simulation one can define samples of LG 
systems based upon the next condition


%.........................................................................
%Additional criteria
\begin{enumerate}
\item[(v)]{The Local Group pair must be located in the right environment 
with respect to the XX Cluster.}
\end{enumerate}
%.........................................................................



Due to the intrinsic nature of constrained simulations, we expect that 
these LG samples be the most faithful systems that resemble the properties
and the environment of our LG. In fact, the three CLUES simulations appear 
to present a common T-web/V-web environment for LG systems, such as shown 
in the figure \ref{fig:LG_CLUES_Environment}. Based in this we propose a 
new criteria of selection to define more realistic LG systems in the
unconstrained simulation, this consists in tanking the extreme values of 
each V-web/T-web eigenvalue associated to the host environment of the three 
LG systems in CLUES. With this range established (see table 
\ref{Tab:Lambdas_LG}) we filter the IP sample in Bolshoi simulation in 
order to build the samples Constructed Local Groups V-web based (CLGV) 
and T-web based (CLGT). Finally in the table \ref{Tab:Samples_Size} we 
show the sizes of all defined samples for each simulation.



%*************************************************************************
\section{Finding a Local Group environment}
\label{sec:experiments}
%*************************************************************************



In this section we prepare some numerical experiments with the halos 
samples and their environment, with special emphasis on the isolated pairs 
and LG samples. All this in order to find for environment correlations and 
common properties between LG systems.


%-------------------------------------------------------------------------
\subsection{Comparison of the two simulations}
\label{subsec:comparison_simulations}
%-------------------------------------------------------------------------


Prior to study of isolated and LG samples and to determinate possible 
correlations between their properties, it is necessary to establish the 
equivalence between all simulations that we will use, with the aim to 
eliminate effects due to construction process of each one.


In first place, we analyse the mass distribution of individual halos and 
in next figure we show the integrated mass function (IMF) for halos sample
with $M \geq 1\times 10^{11} M_{\odot}$.


Although in high masses the IMF of each simulation are slightly different, 
in low mass region, where the most halos of our interest are, the IMFs are 
quite similar indicating that the sample of halos in each simulation have 
the same mass distribution, while the small differences are due to finite 
size of samples. Another aspect in the figure \ref{fig:IMF_Halos} is the 
position of LG halos, they are distributed across the mass range of halos 
that we have set (see in section \ref{section:Def_Samples}), indicating 
that there is not an apparent condition with their individual masses.



The situation is quite different respect to mass ratio index of isolated 
pairs sample (MRI). In the figure \ref{fig:Index_Pairs} (a) we show the 
integrated distribution of MRI, where the approximately linear behaviour 
indicates an uniformly distribution. The interesting aspect here is the 
closeness between the LG sample values, indicating a possible common 
property in these systems, even though this could be established a priori 
by construction. In the \ref{fig:Index_Pairs} (b) is plotted the 
integrated distribution of total pair mass, and again like the IMF, there 
is not a preferential value.



Once established the concordance between the defined samples, we proceed 
to analyse the distribution of the eigenvalue of shear velocity tensor 
\ref{eq:V_web}, for this we assume that the halos are good tracers of the 
environment properties and therefore we evaluate each eigenvalue in the 
center mass of each halo, mapping of this way the complete distribution 
(figure \ref{fig:lambda_histogram}). The black curves correspond to the 
unconstrained simulation (Bolshoi) and the color curves to the constrained 
simulations (CLUES), aditionally we calculate the cosmic variance, showed 
in purple curves, dividing the Bolshoi volume in smaller parts with a size 
comparable to the CLUES volume ($64 h^{-1 }$ Mpc of side). What is 
interesting here is the clear difference between the eigenvalues 
distributions of the both types of simulations, inclusively the 
constrained distributions does not match between the cosmic variance, 
indicating that although both simulations were made with the same cosmology 
(see subsections \ref{subsec:CLUES_simulations} and 
\ref{subsec:Bolshoi_simulation}), the constrained simulations have a 
significantly different environment properties (spatial matter distribution) 
compared to the average expected from a random volume with the same 
comoving size.


%-------------------------------------------------------------------------
\subsection{Defining the LG environment}
\label{subsec:LG_construction}
%-------------------------------------------------------------------------


With the aim of constructing a sample of LG systems in Bolshoi simulations, 
we use the eigenvalues of the three LG systems in each CLUES simulation 
and choose a range for each one based in the extreme values of the six 
halos, this with the expectation of reproducing the LG specific conditions 
in constrained simulations. This criteria can be thought as a first 
approximation to establish a more faithful sub-sample into the isolated 
pairs. In table \ref{Tab:Lambdas_LG} we show the used extreme values for 
each eigenvalue of local shear tensor.



As a self-consistency test, we apply this criteria to CLUES simulations 
and find, on average, three LG systems in each one. But to avoid confusion, 
we keep defining the LG sample as the three initial pairs. Finally, we 
construct the LG sample of Bolshoi, where the sample size is illustrated 
in Table \ref{Tab:Samples}.



Once we have established the equivalence of samples in each simulation and 
have defined the LG samples, we proceed to calculate correlations between 
halo samples and their environment. As was mentioned in section 
\ref{sec:Vweb}, we do not use an specific value of eigenvalue threshold 
$\lambda_{th}$, instead of this, we explore a relatively wide range of 
this parameter (i.e. $0 \leq \lambda_{th} \leq 1$) and calculate 
distributions respect to each eigenvalue individually.



At first place, we calculate the environment of each LG system in CLUES 
simulations, for this we use the $\lambda_{th}$ scheme to classify it in 
void, sheet, filament or knot (see \ref{sec:Vweb}), with $\lambda_{th}$
into the threshold range. Figure \ref{fig:LG_Env_CLUES} is obtained.



At first place, we calculate the integrated distribution of each 
eigenvalue for individual halos, isolated pairs and LG samples. For this, 
we associate to each halo a value of environment (set of eigenvalues)
according to its center of mass position in a smoothed grid ($256^3$ 
cells for Bolshoi and $64^3$ for CLUES, or equivalently a resolution of 
$1.0 \mbox{Mpc h}^{-1} $ per cell, according to the physical size of the 
pairs.)



%*************************************************************************
\section{Results}
\label{sec:Results}
%*************************************************************************

%-------------------------------------------------------------------------
\subsection{Bias induced on kinematic and dynamics properties}
\label{subsec:bias_kinematic}
%-------------------------------------------------------------------------

... Total Mass

... Radial vs. tangential velocities

... Angular Momentum

... Total Mechanical energy.

... Reduced Spin


%-------------------------------------------------------------------------
\subsection{Bias induced on the Mass Assembly Histories}
\label{subsec:bias_MAH}
%-------------------------------------------------------------------------

... Last major merger. Formation Time. Assembly Time.

%-------------------------------------------------------------------------
\subsection{Pair Alignment with the Cosmic Web}
\label{subsec:alignment_cosmic_web}
%-------------------------------------------------------------------------

... Separation.

... Relative velocity.

... Angular Momentum.


%*************************************************************************
\section{Conclusions}
\label{sec:conclusions}
%*************************************************************************


%*************************************************************************
\section*{Acknowledgments}  
%*************************************************************************


\bibliographystyle{mn2e}
\bibliography{references} 




%*************************************************************************
%FIGURES AND TABLES
%*************************************************************************


%.........................................................................
%Mean density for each environment
\begin{flushleft}
\begin{figure*}
\begin{center}

  \includegraphics[keepaspectratio=true,width=0.6\textheight]
  {./figures/cell_types_density.pdf}

  \captionof{figure}{\small Mean density parameter for each of the 
  defined environments according to the chosen $\lambda_{th}$ value and 
  for both classification schemes.}

  \label{fig:mean_density}
  \vspace{0.1 cm}

\end{center}
\end{figure*}
\end{flushleft}
%.........................................................................


%.........................................................................
%Visual impresion
\begin{flushleft}
\begin{figure*}
\begin{center}

  \includegraphics[trim = 40mm 12mm 37mm 10mm, clip, keepaspectratio=true,
  width=0.7\textheight]{./figures/cosmicweb_visual_Tweb.pdf}
  \includegraphics[trim = 40mm 12mm 37mm 10mm, clip, keepaspectratio=true,
  width=0.7\textheight]{./figures/cosmicweb_visual_Vweb.pdf}

  \captionof{figure}{\small Visual impression of the density field (left),
  and of each classification scheme with the $\lambda_{th}$ values obtained 
  by our criteria (others). Our color convention for each environment is 
  (white) - void, (light gray) - sheet, (gray) - filament, (black) - knot. }

  \label{fig:visual_impression}
  \vspace{0.1 cm}

\end{center}
\end{figure*}
\end{flushleft}
%.........................................................................


%.........................................................................
%Percolation analysis
\begin{flushleft}
\begin{figure*}
\begin{center}

  \includegraphics[trim = 5mm 10mm 15mm 18mm, clip, keepaspectratio=true,
  width=0.35\textheight]{./figures/voids_regions_percolation.pdf}

  \includegraphics[trim = 0mm 00mm 00mm 00mm, clip, keepaspectratio=true,
  width=0.35\textheight]{./figures/voids_regions_volume.pdf}

  \captionof{figure}{\small Percolation analysis of void regions for 
  different $\lambda_{th}$ values and for both defined classification 
  schemes.}

  \label{fig:percolation_analysis}
  \vspace{0.1 cm}

\end{center}
\end{figure*}
\end{flushleft}
%.........................................................................


%.........................................................................
%Distribution of eigenvalues of the inertia tensor
\begin{flushleft}
\begin{figure*}
\begin{center}

  \includegraphics[trim = 7mm 10mm 1mm 0mm, clip, keepaspectratio=true,
  width=0.36\textheight]{./figures/voids_inertia_tensor_Tweb}
  \includegraphics[trim = 7mm 10mm 1mm 0mm, clip, keepaspectratio=true,
  width=0.36\textheight]{./figures/voids_inertia_tensor_Vweb}

  \captionof{figure}{\small Percolation analysis of void regions for 
  different $\lambda_{th}$ values and for both defined classification 
  schemes.}

  \label{fig:distro_inertia}
  \vspace{0.1 cm}

\end{center}
\end{figure*}
\end{flushleft}
%.........................................................................


%.........................................................................
%Table of extreme values of LG environment
\begin{flushleft}
\begin{table*}
\begin{center}

  \begin{tabular}{l | c c c} \hline
	& $\bds{\lambda_{1}}\ [10^{-1}]$ & $\bds{\lambda_{2}}\ [10^{-1}]$  & $\bds{\lambda_{3}}\ [10^{-1}]$ \\ \hline
	\textbf{Minim value} & 1.78 & -6.29$\times 10^{-2}$ & -1.98 \\
	\textbf{Maxim value} & 3.49 & 1.21 & -8.85$\times 10^{-1}$ \\ \hline
  \end{tabular}
  
  \captionof{table}{\small Extreme values for each V-web eigenvalues to 
  construct LG samples.}
  
  \label{Tab:Lambdas_LG}
  
\end{center}
\end{table*}
\end{flushleft}
%.........................................................................


%.........................................................................
%IMF halos
\begin{flushleft}
\begin{figure*}
\begin{center}

  \includegraphics[keepaspectratio=true,width=0.35\textheight]
  {./figures/integrated_mass_halos.pdf}

  \captionof{figure}{\small Integrated mass function of individual halos 
  of Bolshoi.}

  \label{fig:IMF_Bolshoi}
  \vspace{0.1 cm}

\end{center}
\end{figure*}
\end{flushleft}
%.........................................................................


\end{document}
